\begin{problem}{來用程式計算程式的時間複雜度吧!(Time complexity)}{standard input}{standard output}{1 second}{256 megabytes}

還記得我們有教過時間複雜度怎麼估計嗎?最簡單的方式就是計算迴圈總共跑了幾次。

那現在給你一段擁有 $n$ 行的部分程式碼,每行只會包含 for 迴圈,格式為 for(int 變數名稱A=0;變數名稱A<變數名稱B;變數名稱A++),請你設計一個程式計算這一個程式碼的時間複雜度是多少,為了統一格式,請將完整的時間複雜度輸出,運算式左到右依序是迴圈上到下。

\textbf{運算式乘號請省略掉,但加號不可省略。}

\textbf{兩層以上的迴圈會在每一行的前面空 $4$ 個空白區分,詳細的例子請參考範例輸入。}

\InputFile
只有一組資料。

第一行輸入一個正整數 $n$。

接下來輸入 $n$ 行,每行包含一個字串,保證格式滿足題目敘述。

\textbf{測資範圍限制}

\begin{itemize}

\item $ 1 \le n \le 10$。

\item 保證變數名稱都只會有一個字元,且這一個字元一定是小寫英文字母。

\item 每一個迴圈內最多只會再有一個迴圈 (意旨不會出現同一層有兩個以上的迴圈的情況)。

\end{itemize}

\OutputFile
輸出一個字串,表示其時間複雜度,格式為 $O($運算式$)$。

\Scoring
以下為本題的配分,本題的滿分為 $100$ 分,只要你的程式通過某個子任務就可以拿到該子任務的分數。

\begin{center}
  \begin{tabular}{ | c | c | c | c | } \hline
    \bf{子任務} &
    \bf{條件限制} &
    \bf{分數} &
    \bf{附加限制} \\ \hline
    $1$ & 題目範例 & $1$ & 無 \\ \hline
    $2$ & $n = 1$ & $23$ & 無 \\ \hline
    $3$ & 沒有兩層以上的迴圈出現 & $25$ & 無 \\ \hline
    $4$ & 題目範圍限制 & $51$ & 須通過子任務 1、2、3 \\ \hline
    \end{tabular}
\end{center}

\Examples

\begin{example}
\exmpfile{example.01}{example.01.a}%
\exmpfile{example.02}{example.02.a}%
\exmpfile{example.03}{example.03.a}%
\exmpfile{example.04}{example.04.a}%
\end{example}

\end{problem}

