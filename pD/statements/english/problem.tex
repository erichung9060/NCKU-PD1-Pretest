\begin{problem}{隊伍調查}{standard input}{standard output}{1 second}{256 megabytes}

台灣人最愛做的事情就是排隊了,觀察到這個現象的助教們想要來做一個酷酷的調查。

整個排隊隊伍會有 $n$ 個人,一開始排隊隊伍由左到右依序是 $1 \sim n$,但是有些人不太守規矩因此常常發生插隊的狀況。

這邊定義如果 $a$ 插隊 $b$,表示 $a$ 跑去排到了 $b$ 的左邊,舉例來說如果隊伍目前是 $\{1,2,3,4,5\}$,假如現在 $2$ 插隊 $4$,隊伍就會變成 $\{1,3,2,4,5\}$,雖然你一定很好奇為什麼要插後面的隊,但先不要問這麼多。

助教們要做的調查必須拜託你設計一個有趣的程式,這個程式要能支援三種操作。

操作 1:輸入兩個正整數 $a,b$,表示現在發生了一個插隊事件,$a$ 插隊 $b$。

操作 2:輸入一個正整數 $p$,表示現在要查詢隊伍由左邊數來第 $p$ 個人是誰。

操作 3:結束程式。

相信聰明的你已經想到要怎麼做了吧!請你設計一個可以滿足題目要求的程式出來吧!

\InputFile
第一行輸入一個正整數 $n$。

接下來輸入數行,每行的第一個數字表示操作的類型,再依據操作的類型輸入相對應數量的參數。

保證 $n$ 的大小不超過 $2 \times 10^5$,且操作 $1,2$ 輸入的參數都會介於 $1 \sim n$ 之間,且操作 $2$ 的次數不超過 $100$ 次。

全部的操作次數加起來不會超過 $5000$ 次,且保證操作 $3$ 一定只會出現在最後一個操作。

另外保證不會有自己插隊自己的情況發生。

\OutputFile
當執行操作 $2$ 後,根據參數輸出相對應的答案,輸出完要換行。

\Example

\begin{example}
\exmpfile{example.01}{example.01.a}%
\end{example}

\Note
提示:你可以用兩個 Linked-List 或陣列紀錄某個人的前面那個人跟後面那個人是誰,然後再根據插隊發生的改變對陣列做調整。

如果你發現你的程式會超過時間限制,表示你的程式的效率 (時間複雜度) 比正常的做法差太多了。

\end{problem}

